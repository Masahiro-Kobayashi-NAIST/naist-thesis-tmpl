\usepackage[dvipdfmx]{graphicx}
\usepackage[utf8]{inputenc}
\usepackage[T1]{fontenc}
\usepackage{lmodern}
\usepackage{amsmath,amssymb,amsthm}
\let\equation\gather
\let\endequation\endgather
\usepackage{fancybox}
\usepackage[flushmargin,symbol]{footmisc}
\usepackage[nottoc]{tocbibind}
\usepackage[dvipdfmx,%
 bookmarks=true,%
 bookmarksnumbered=true,%
 setpagesize=false,%
 colorlinks=false,%
 linkbordercolor={0.8 0.8 0.8},%
 citebordercolor={0.8 0.8 0.8},%
 pdfborder={0 0 0.6},%
% urlcolor=black,linkcolor=black,citecolor=black,%
 pdftitle={},% 修論のタイトルを入れる
 pdfauthor={XX XX},% 名前を入れる
 pdfsubject={Master's thesis},%
 pdfkeywords={\ekeywords}]{hyperref}
\usepackage{pxjahyper}
\usepackage{bxwareki}

% page layout from NAIST style
%
\textheight=20.6truecm                % 高さ
\textwidth=14.5truecm                 % 横幅 (約36文字)
\oddsidemargin=0.6truecm              % 左の空きの幅
\evensidemargin=-3.8truecm            % 右の空きの幅

% 第n章を n. 表記に
\renewcommand{\prechaptername}{}     % 数字の前の文字
\renewcommand{\postchaptername}{.}   % 数字の後ろの文字

% 数式の途中での改行を許可
\allowdisplaybreaks[1]

\newtheoremstyle{mydef}%スタイルの名前
{\topsep}%(このスタイルの)定理型環境の上にはいる空白
{\topsep}%同じ様に,下にはいる空白
{\normalfont}%本文のフォント
%{2pt}%見出し(「定理」とか「補題」とか)の前にはいる空白
{}%見出し(「定理」とか「補題」とか)の前にはいる空白
{\bfseries}%見出しの書体.ここではHelveticaでボールド.
{\bfseries}%見出しの後にはいる句点("."を入れたければここに"."を書く)
{0.8em}%見出しの後にはいるスペース
{}%

\theoremstyle{mydef}%
\newtheorem{theorem}{定理\!}[section]
\newtheorem{definition}{定義\!}[section]
\newtheorem{lemma}{補題\!}[section]
\newtheorem{corollary}{系\!}[section]
\newtheorem{example}{例\!}[section]

\newcommand{\figlab}[1]{\label{fig:#1}}
\newcommand{\figref}[1]{図\!~\ref{fig:#1}} % 図○
\newcommand{\forlab}[1]{\label{for:#1}}
\newcommand{\forref}[1]{式\!~(\ref{for:#1})} % 式
\newcommand{\seclab}[1]{\label{sec:#1}}
\newcommand{\secref}[1]{\ref{sec:#1}\!~節} % 節
\newcommand{\chaplab}[1]{\label{chap:#1}}
\newcommand{\chapref}[1]{\ref{chap:#1}\!~章} % 章
\newcommand{\exlab}[1]{\label{ex:#1}}
\newcommand{\exref}[1]{{\bfseries 例\!~\ref{ex:#1}}} % 例
\newcommand{\lmmlab}[1]{\label{lmm:#1}}
\newcommand{\lmmref}[1]{{\bfseries 補題\!~\ref{lmm:#1}}} % 補題
\newcommand{\lmmrefm}[1]{補題\!~\ref{lmm:#1}} % 補題
\newcommand{\thmlab}[1]{\label{thm:#1}}
\newcommand{\thmref}[1]{{\bfseries 定理\!~\ref{thm:#1}}} % 定理
\newcommand{\thmrefm}[1]{定理\!~\ref{thm:#1}} % 定理
\newcommand{\corlab}[1]{\label{cor:#1}}
\newcommand{\corref}[1]{{\bfseries 系\!~\ref{cor:#1}}} % 系
\newcommand{\steplab}[1]{\label{step:#1}}
\newcommand{\stepref}[1]{{\bfseries 手順\!~\ref{step:#1}}} % 手順



%%%%%%%%%%%%%%%%%%%%%%%%%%%%%%%%%%%%%%%%%%%%%%%%%%%%%%%%%%%%%
%
% Class Setting
%
%%%%%%%%%%%%%%%%%%%%%%%%%%%%%%%%%%%%%%%%%%%%%%%%%%%%%%%%%%%%%

\def\thebibliography#1{\chapter*{参考文献\markboth
 {参 考 文 献}{参 考 文 献}\addcontentsline{toc}{chapter}{参考文献}} \list
 {[\arabic{enumi}]}{\settowidth\labelwidth{[#1]}\leftmargin\labelwidth
 \advance\leftmargin\labelsep
 \usecounter{enumi}}
 \def\newblock{\hskip .11em plus .33em minus -.07em}
 \sloppy
 \sfcode`\.=1000\relax}
\let\endthebibliography=\endlist

%chapterのフォントサイズ変更
\makeatletter%%
\def\@makechapterhead#1{\hbox{}%
  \vskip-1\Cvs
  {\parindent\z@
%  \reset@font\LARGE\bfseries
   \raggedright\reset@font\Large\bfseries% 左揃え
   \ifnum \c@secnumdepth >\m@ne
     \setlength\@tempdima{\linewidth}%
     \vtop{\hsize\@tempdima%
         \@chapapp\thechapter\@chappos\mbox{\ \ }%
     #1}%
   \else
     #1\relax
   \fi}\nobreak\vskip1\Cvs}
\makeatother%%

%schapterのフォントサイズ変更
\makeatletter%%
\def\@makeschapterhead#1{\hbox{}%
  \vskip-1\Cvs
  {\parindent \z@ \raggedright
    \normalfont
    \interlinepenalty\@M
    \Large\headfont #1\par\nobreak
    \vskip1\Cvs}}
\makeatother%%

%sectionのフォントサイズ変更
\makeatletter%%
\renewcommand{\section}{%
  \@startsection{section}% #1 見出し
   {1}% #2 見出しのレベル
   {\z@}% #3 横組みの場合,見出し左の空き(インデント量)
   {1.5\Cvs \@plus.5\Cdp \@minus.2\Cdp}% #4 見出し上の空き
   {.5\Cvs \@plus.3\Cdp}% #5 見出し下の空き (負の値なら見出し後の空き)
  {\raggedright\reset@font\large\bfseries}% 左揃え
}%
\makeatother%%

%subsectionのフォントサイズ変更
\makeatletter%%
\renewcommand{\subsection}{%
  \@startsection{subsection}% #1 見出し
   {1}% #2 見出しのレベル
   {\z@}% #3 横組みの場合,見出し左の空き(インデント量)
   {1.5\Cvs \@plus.5\Cdp \@minus.2\Cdp}% #4 見出し上の空き
   {.5\Cvs \@plus.3\Cdp}% #5 見出し下の空き (負の値なら見出し後の空き)
  {\raggedright\reset@font\normalsize\bfseries}% 左揃え
}%
\makeatother%%

% ページ冒頭に来る表題の上のスペースを消す
% http://blog.livedoor.jp/green_idea/archives/1464846.html
\makeatletter
\def\@startsection#1#2#3#4#5#6{%
  \if@noskipsec \leavevmode \fi
  \par
  \@tempskipa #4\relax
  \if@english \@afterindentfalse \else \@afterindenttrue \fi
  \ifdim \@tempskipa <\z@
    \@tempskipa -\@tempskipa \@afterindentfalse
  \fi
  \if@nobreak
    \everypar{}%
  \else
    \addpenalty\@secpenalty
    \ifdim \@tempskipa >\z@
      \vskip\@tempskipa
      \if@slide\else
        \null
        \vspace{-\baselineskip}%
      \fi
    \fi
  \fi
  \noindent
  \@ifstar
    {\@ssect{#3}{#4}{#5}{#6}}%
    {\@dblarg{\@sect{#1}{#2}{#3}{#4}{#5}{#6}}}}
\makeatother

%式番号に節を加える
\makeatletter
  \renewcommand{\theequation}{%
  \thesection.\arabic{equation}}
    \@addtoreset{equation}{section}
\makeatother

%図にも節番号を加える
%\makeatletter
%  \renewcommand{\thefigure}{%
%  \thesection.\arabic{figure}}
%  \@addtoreset{figure}{section}
%\makeatother

%図に章番号を加える
\makeatletter
 \renewcommand{\thefigure}{%
  \thechapter.\arabic{figure}}
   \@addtoreset{figure}{chapter}
 \makeatother
\makeatletter

% jsbook 用.各章の扉を jsarticle と同じ見た目にする
\makeatletter
\renewcommand{\chapter}{%
 \if@openright\cleardoublepage\else\clearpage\fi
 \global\@topnum\z@
 \secdef\@chapter\@schapter}
\makeatother

% appendix追加
\renewcommand{\appendix}{\par
        \setcounter{section}{0}
        \setcounter{subsection}{0}
        \def\thesection{\Alph{section}.}
        \def\thesubsection{\Alph{section}.\arabic{subsection}}
        \def\thesubsubsection{\Alph{section}.\arabic{subsection}.\arabic{subsubsection}}
        \newpage
}

% ページ数を下部に表示 (reportならいらない)
%\makeatletter
%\def\ps@plainfoot{%
%  \let\@mkboth\@gobbletwo
%  \let\@oddhead\@empty
%  \def\@oddfoot{\normalfont\hfil-- \thepage\ --\hfil}%
%  \let\@evenhead\@empty
%  \let\@evenfoot\@oddfoot}
%\let\ps@plain\ps@plainfoot
%\makeatother
%\pagestyle{plain}

% フッタまわりのスペースの設定 (reportならいらない)
%\setlength\footskip{2\baselineskip}
%\addtolength{\textheight}{-2\baselineskip}

% 左右のマージンを等しくする (reportならいらない)
%\setlength{\evensidemargin}{\oddsidemargin}

% 目次に subsection まで表示する
\setcounter{tocdepth}{4}
